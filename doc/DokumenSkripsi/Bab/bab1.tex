%versi 2 (8-10-2016) 
\chapter{Pendahuluan}
\label{chap:intro}
   
\section{Latar Belakang}
\label{sec:label}

Salah satu dampak kemajuan teknologi adalah mudahanya mendapat informasi yang diinginkan oleh seseorang. Informasi tersebut dapat diperoleh dari media cetak dan juga media elektronik. Banyaknya jumlah informasi yang diperoleh seringkali menghambat proses pengambilan keputusan, karena banyak hal yang harus dipertimbangkan. Pengambilan keputusan merupakan suatu tindakan memilih suatu alternatif terbaik dari banyak alternatif lain. Salah satu persoalan yang ada saat ini adalah sulitnya menentukan pilihan dalam memilih rumah.

Rumah merupakan salah satu kebutuhan pokok manusia yang berfungsi sebagai tempat untuk berlindung dan beristirahat. Seiring dengan bertambahanya jumlah penduduk, kebutuhan akan tempat tinggal semakin meningkat. Hal ini tentu membuat rumah menjadi salah satu aset yang berharga. Mengingat pentingnya rumah bagi seseorang saat ini, tentu tidak boleh sembarangan dalam menentukan pilihan ketika membeli rumah. Pengambilan keputusan untuk menentukan pilihan rumah bukanlah hal yang mudah karena ada banyak alternatif pilihan yang ada.

Selain banyaknya alternatif pilihan, ada banyak kriteria yang harus diperhatikan dalam pemilihan rumah. Kriteria tersebut dapat dikelompokkan menjadi faktor subjektif dan faktor objektif.
Faktor subjektif merupakan kriteria yang nilainya bersifat kualitatif atau tidak mempunyai nilai numerik, dan faktor objektif yang nilainya bersifat kuantitatif atau mempunyai nilai numerik.
Banyaknya jumlah alternatif pilihan dan jumlah kriteria yang ada menimbulkan suatu permasalahan bagi pembeli dalam pengambilan keputusan. Oleh karena itu dibutuhkan suatu metode yang dapat mempermudah pembeli dalam pengambilan keputusan multi-atribut dengan mempertimbangkan faktor subjektif dan faktor objektif. 

Berdasarkan masalah yang ada, maka dibutuhkan suatu sistem yang dapat membantu seseorang untuk mengambil keputusan dengan cepat dan akurat. Sistem pendukung keputusan merupakan salah satu solusi untuk menyelesaikan permasalahan ini. Sistem pendukung keputusan yang dibangun untuk menyelesaikan permasalah ini akan menggunakan metode Brown Gibson. Metode Brown Gibson merupakan metode matematis yang dapat menganalisis alternatif-alternatif solusi berdasarkan konsep \textit{Preferences of Measurement}, yang mengkombinasikan faktor-faktor objektif dan faktor-faktor subjektif.



\section{Rumusan Masalah}
\label{sec:rumusan}
\begin{enumerate}
	\item Bagaimana menganalisis kriteria-kriteria yang menjadi pertimbangan dalam pemilihan rumah?
	\item Bagaimana cara kerja metode Brown Gibson dalam mendukung keputusan pemilihan rumah?
	\item Bagaimana membangun perangkat lunak pengambilan keputusan pembelian rumah berbasis metode Brown Gibson?
\end{enumerate}

\section{Tujuan}
\label{sec:tujuan}
\begin{enumerate}
	\item Menganalisis dan menentukan kriteria-kriteria dalam pemilihan rumah.
	\item Memahami cara kerja metode Brown Gibson dalam mendukung pengambilan keputusan pembelian rumah.
	\item Membangun perangkat lunak pendukung keputusan berbasis metode Brown Gibson untuk pembelian rumah.
\end{enumerate}


\section{Batasan Masalah}
\label{sec:batasan}
Untuk mempermudah pembuatan template ini, tentu ada hal-hal yang harus dibatasi, misalnya saja bahwa template ini bukan berupa style \LaTeX{} pada umumnya (dengan alasannya karena belum mampu jika diminta membuat seperti itu)

\dtext{8}

\section{Metodologi}
\label{sec:metlit}
Tentunya akan diisi dengan metodologi yang serius sehingga templatenya terkesan lebih serius.

\dtext{9}

\section{Sistematika Pembahasan}
\label{sec:sispem}
Rencananya Bab 2 akan berisi petunjuk penggunaan template dan dasar-dasar \LaTeX.
Mungkin bab 3,4,5 dapt diisi oleh ketiga jurusan, misalnya peraturan dasar skripsi atau pedoman penulisan, tentu jika berkenan.
Bab 6 akan diisi dengan kesimpulan, bahwa membuat template ini ternyata sungguh menghabiskan banyak waktu.

\dtext{10}